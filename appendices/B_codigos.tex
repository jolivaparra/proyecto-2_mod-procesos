\chapter{Códigos Fuente MATLAB}

A continuación se presenta el código de todos los archivos utilizados para la simulación y creación de los gráficos del proyecto.

Este se divide en 4 archivos, cada uno cumpliendo una función específica la cual se detalla a continuación:

\begin{itemize}
    \item \textbf{main.m} - Archivo principal que orquesta la simulación, llama a las funciones necesarias y genera los gráficos.
    \item \textbf{parametros.m} - Define todos los parámetros físicos y de diseño utilizados para las simulaciones.
    \item \textbf{Entradas.m} - Archivo de tipo clase que contiene todas los escenarios de perturbaciones utilizados en todas las simulaciones (Día Despejado, Nublado e Intermitente).
    \item \textbf{graficos.m} - Función encargada de generar y personalizar todos los gráficos resultantes de las simulaciones en los distintos escenarios de perturbaciones.
    \item \textbf{modelo.m} - Función que contiene el modelo de EDO's del sistema que se resuelve numerosas veces a lo largo del proyecto.
\end{itemize}

A continuación se inserta todo el código utilizado.
\newpage

\begin{multicols}{2}
    \setstretch{0.95}
    \small

    \lstinputlisting[caption={Código principal de ejecución}]{code/main.m}

    \lstinputlisting[caption={Generación de gráficos personalizados}]{code/modelo.m}

    \lstinputlisting[caption={Definición de parámetros físicos y de diseño}]{code/parametros.m}

    \lstinputlisting[caption={Definición de escenarios de perturbaciones}]{code/Entradas.m}

    \lstinputlisting[caption={Generación de gráficos personalizados}]{code/graficos.m}

\end{multicols}