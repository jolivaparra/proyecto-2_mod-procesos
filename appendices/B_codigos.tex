<<<<<<< HEAD:report/appendices/B_codigos.tex
\chapter{Códigos Fuente MATLAB}

A continuación se adjuntan los scripts utilizados para la simulación. Se incluyen las rutinas de definición de parámetros, modelo dinámico, generación de perturbaciones y el script de ejecución principal.

% Iniciamos el entorno de dos columnas
\begin{multicols}{2}

    % --- ARCHIVO 1: MAIN ---
    \section{Script Principal (main.m)}
    % Ajusta la ruta 'codigos/main.m' a donde tengas tu archivo realmente
    \lstinputlisting[caption={Ejecución y lógica principal}, label={code:main}]{main.m}

    % --- ARCHIVO 2: MODELO ---
    \section{Modelo Dinámico (modelo.m)}
    \lstinputlisting[caption={Ecuaciones diferenciales del sistema}, label={code:modelo}]{modelo.m}

    % --- ARCHIVO 3: PARÁMETROS ---
    \section{Parámetros (parametros.m)}
    \lstinputlisting[caption={Definición de constantes físicas}, label={code:params}]{parametros.m}

    % --- ARCHIVO 4: ENTRADAS ---
    \section{Perturbaciones (Entradas.m)}
    \lstinputlisting[caption={Generación de perfiles climáticos}, label={code:entradas}]{Entradas.m}

    % --- ARCHIVO 5: GRÁFICOS ---
    \section{Generación de Gráficos (graficos.m)}
    \lstinputlisting[caption={Funciones de visualización}, label={code:graficos}]{graficos.m}

\end{multicols}
=======
\chapter{Códigos Fuente en MATLAB} \label{app:codigos}

A continuación se adjunta la implementación computacional completa del proyecto. Los códigos han sido desarrollados y ejecutados en MATLAB R2025b. Se incluyen los archivos de definición de parámetros, modelo dinámico (EDO), generación de perturbaciones y el script principal de ejecución y graficado.

% Iniciamos el entorno de dos columnas para ahorrar espacio
>>>>>>> feat/app-code:appendices/B_codigos.tex
