\chapter{Introducción}

En la etapa anterior de este trabajo \cite{proyecto1}, se completó el desarrollo del modelado matemático que describe la dinámica térmica de un sistema de enfriamiento por agua en circuito cerrado, acoplado a un panel solar fotovoltaico. El sistema físico modelado consta de dos elementos activos manipulables: una bomba hidráulica encargada de la circulación del fluido refrigerante y un ventilador integrado a un intercambiador de calor, diseñado para disipar la energía térmica hacia el ambiente.

El resultado de dicha etapa fue un sistema de Ecuaciones Diferenciales Ordinarias (EDO) validado, que relaciona las entradas de tensión ($V_\text{bomb}, V_\text{vent}$) y las perturbaciones ambientales con la evolución temporal de la temperatura de operación del panel ($T_p$).

\section*{Planteamiento del Problema}

La eficiencia de conversión eléctrica de los paneles fotovoltaicos es sensible a la temperatura, cada grado sobre su temperatura de operación (\SI{25}{\degreeCelsius}) disminuye su rendimiento ligeramente. Sin un mecanismo de regulación activo, el panel queda expuesto a las variaciones estocásticas de la irradiancia solar y la temperatura ambiente.

El análisis preliminar del sistema (simulaciones en lazo abierto sin refrigeración) revela que, bajo condiciones de alta carga térmica, el panel tiende a alcanzar temperaturas de equilibrio cercanas a los \SI{71}{\degreeCelsius}. Este comportamiento supera ampliamente su estandar de operación, conllevando dos riesgos:
\begin{enumerate}
    \item Degradación acelerada de los materiales por estrés térmico.
    \item Pérdida de potencia útil generada.
\end{enumerate}

El desafío que aborda el presente proyecto es que la operación manual o fija del sistema de refrigeración es insuficiente. Operar los actuadores a máxima potencia de forma constante garantiza el enfriamiento, pero resulta \textbf{energéticamente ineficiente}, consumiendo gran parte de la energía que el propio panel genera. Por consiguiente, se hace indispensable diseñar un sistema de control en \textbf{lazo cerrado} que gestione este compromiso de forma autónoma.

\section*{Variables del Sistema de Control}

Para el diseño del esquema de control, se definen las siguientes variables sobre el modelo dinámico:

\begin{itemize}
    \item \textbf{Variable Controlada (Salida):} Temperatura superficial del panel solar ($T_p$).
    \item \textbf{Variables Manipuladas (Entradas):} Tensión aplicada a la bomba ($V_\text{bomb}$) y al ventilador ($V_\text{vent}$).
    \item \textbf{Perturbaciones:} Irradiancia solar ($G$), Temperatura ambiente ($T_\text{amb}$) y Velocidad del viento ($v_\text{vent}$).
\end{itemize}

\section*{Objetivos del Proyecto}

\subsection*{Objetivo General}
Diseñar e implementar un sistema de control realimentado (Controlador Proporcional) sobre el modelo dinámico del panel solar, capaz de satisfacer las especificaciones de respuesta transitoria y estacionaria definidas para la preservación del equipo y la eficiencia energética.

\subsection*{Especificaciones de Respuesta}
De acuerdo con los criterios de diseño postulados en la etapa de modelado, el controlador debe cumplir estrictamente con las siguientes especificaciones de desempeño:

\begin{enumerate}
    \item \textbf{Límite de Seguridad Térmica:} Mantener la temperatura superficial del panel por debajo de \SI{55}{\degreeCelsius} durante las horas de máxima irradiancia, con el fin de minimizar la degradación de los materiales.
    \item \textbf{Estabilidad del Régimen:} Garantizar un régimen térmico estable, evitando oscilaciones superiores a \SI{2}{\degreeCelsius} en la superficie del panel para prevenir fatiga por ciclo térmico.
    \item \textbf{Eficiencia Energética:} Optimizar el compromiso entre rendimiento térmico y consumo energético, mediante estrategias de refrigeración dinámica que ajusten su intensidad estrictamente de acuerdo con las condiciones ambientales.
\end{enumerate}

\subsection*{Objetivos Específicos}
Para dar cumplimiento a las especificaciones anteriores, se plantean las siguientes tareas:
\begin{itemize}
    \item Analizar el comportamiento dinámico en \textbf{lazo abierto} (ante entradas tipo escalón) para establecer la línea base de temperatura sin intervención.
    \item Programar un algoritmo de control Proporcional (P) a aplicar a las entradas manipulables.
    \item Sintonizar la ganancia ($K_c$) y el punto de operación (Offset) mediante variaciones de esta misma con el objetivo de reducir oscilaciones y asegurar el límite térmico.
    \item Evaluar cuantitativamente el desempeño del controlador mediante métricas de respuesta al escalón de referencia (sobrepaso, tiempo de estabilización y error de estado estacionario).
    \item Verificar la robustez del sistema simulando su comportamiento ante perfiles de perturbaciones realistas (días nublados e intermitentes).
\end{itemize}
