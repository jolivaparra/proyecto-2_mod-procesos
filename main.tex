\documentclass[12pt,letterpaper]{report}

% ===========================
% PAQUETES BÁSICOS
% ===========================
\usepackage[spanish]{babel}             % Idioma español
\usepackage[T1]{fontenc}                % Codificación de salida
\usepackage[utf8]{inputenc}             % Codificación de entrada
\usepackage{csquotes}
\usepackage[letterpaper,
    top=2.5cm,
    bottom=2cm,
    left=2.5cm,
    right=2cm,
    headsep=0.3cm,
    footskip=1.2cm]{geometry}
\setlength{\headheight}{14.5pt} % Ya corregido.

\usepackage{newtxtext,newtxmath}        % Fuente Times New Roman
\usepackage{setspace}                   % Interlineado
\usepackage{multicol}

% ===========================
% FIGURAS Y TABLAS
% ===========================
\usepackage{graphicx}                   % Imágenes
\graphicspath{{figures/}}
\usepackage{float}                      % Posicionamiento [H]
\usepackage[table]{xcolor}              % Colores
\usepackage[skip=4pt, labelfont=bf,
            justification=centering,
            margin=2cm]{caption}
\usepackage{subcaption}                 % Leyendas de subfiguras
\usepackage{wrapfig}

\let\Bbbk\relax                % Evitar conflicto con newtxmath
% ===========================
% MATEMÁTICAS Y CIENCIAS
% ===========================
\usepackage{amsmath,amssymb,mathtools}  % Símbolos y entornos matemáticos
\usepackage{siunitx}                    % Unidades físicas

% ===========================
% TÍTULOS Y PÁGINA
% ===========================
\usepackage{titlesec}                   % Formato de capítulos/secciones
\usepackage{fancyhdr}                   % Encabezados y pies de página

% ===========================
% HIPERVÍNCULOS Y CITAS
% ===========================
\usepackage[
    backend=biber,         % Motor de procesamiento recomendado
    style=ieee,            % Estilo de citas numérico (ajustado de ieeetr)
    natbib=true            % Permite usar \citep y \citet
]{biblatex}
\addbibresource{references.bib}

\usepackage[bookmarks=true,
            bookmarksnumbered,
            colorlinks=false,
            hidelinks,
            breaklinks]{hyperref}

\usepackage{listingsutf8} % soporte UTF-8 en listings

% Colores personalizados (opcional)
\definecolor{miblue}{rgb}{0,0,1}
\definecolor{migray}{rgb}{0.4,0.4,0.4}
\definecolor{miteal}{rgb}{0.0,0.5,0.5}

% Configuración general de listings para MATLAB
\lstset{
    language=Matlab,
    inputencoding=utf8,
    extendedchars=true,
    literate={ñ}{{\~n}}1
            {Ñ}{{\~N}}1
            {á}{{\'a}}1
            {é}{{\'e}}1
            {í}{{\'i}}1
            {ó}{{\'o}}1
            {ú}{{\'u}}1
            {Á}{{\'A}}1
            {É}{{\'E}}1
            {Í}{{\'I}}1
            {Ó}{{\'O}}1
            {Ú}{{\'U}}1
            {°}{{$^{\circ}$}}1
            {·}{{$\cdot$}}1,
    backgroundcolor=\color{white},
    basicstyle=\ttfamily\scriptsize,    % tamaño de letra pequeño pero legible
    numbers=left,
    numberstyle=\tiny,
    numbersep=4pt,                      % separación entre números y código
    xleftmargin=0.5em,                  % margen izquierdo del código
    xrightmargin=0.5em,                 % margen derecho del código
    breaklines=true,
    frame=single,
    captionpos=b,
    keywordstyle=\color{miblue},
    commentstyle=\color{migray},
    stringstyle=\color{miteal},
    showstringspaces=false
}

% filas más compactas y fuente más pequeña en todas las tablas
\AtBeginEnvironment{tabular}{\small\renewcommand{\arraystretch}{0.9}}
\AtBeginEnvironment{tabularx}{\small\renewcommand{\arraystretch}{0.9}}
\setlength{\tabcolsep}{4pt}

% INTERLINEADO Y PÁRRAFOS
\onehalfspacing                     % Interlineado 1.5
\setlength{\parskip}{0.4cm}         % espacio entre párrafos
\setlength{\parindent}{1cm}         % sangría

% ENCABEZADOS Y PIE DE PÁGINA
\fancypagestyle{plain}{
    \fancyhf{} % Limpia todos los encabezados y pies existentes
    \fancyfoot[C]{\thepage}
    \renewcommand{\headrulewidth}{0pt} % Asegura que no haya línea en el encabezado
    \renewcommand{\footrulewidth}{0pt} % Asegura que no haya línea en el pie de página
}

% COMANDOS ÚTILES DEL INFORME
\newcommand{\AuthorName}{Joshua S. Oliva Parra}
\newcommand{\ProfessorName}{Juan Pablo Segovia Vera}
\newcommand{\MainTitle}{Modelación de sistema de enfriamiento de agua en
circuito cerrado para paneles fotovoltaicos}

% EXTRAS
\sisetup{
    per-mode = symbol,              % usa "/" en lugar de exponentes negativos
    inter-unit-product = \cdot,     % punto medio para multiplicar unidades
}
\numberwithin{equation}{chapter}
\numberwithin{figure}{chapter}
\numberwithin{table}{chapter}

% === CONFIGURACIÓN DE TÍTULOS DE ÍNDICES (PAQUETE TOCLOFT) ===
\usepackage[titles]{tocloft} 

% 1. Reducir espacio vertical ANTES del título (Sube el título)
\setlength{\cftbeforeloftitleskip}{-2em} 
\setlength{\cftbeforelottitleskip}{-2em}
\setlength{\cftbeforetoctitleskip}{-2em} % También lo aplicamos al índice general si quieres

% 2. Reducir espacio vertical DESPUÉS del título
\setlength{\cftafterloftitleskip}{10pt}
\setlength{\cftafterlottitleskip}{10pt}
\setlength{\cftaftertoctitleskip}{10pt}

% 3. Cambiar tamaño de fuente (De \Huge a \Large)
\renewcommand{\cftloftitlefont}{\Large\bfseries\sffamily}
\renewcommand{\cftlottitlefont}{\Large\bfseries\sffamily}
\renewcommand{\cfttoctitlefont}{\Large\bfseries\sffamily}

\begin{document}

% ========== PORTADA =============
\begin{titlepage}
\begingroup
\linespread{1}\selectfont
\setlength{\parskip}{0pt}

\begin{center}
    % ----- Institución -----
    {\LARGE UNIVERSIDAD DE CONCEPCIÓN} \\[0.3cm]
    {\large FACULTAD DE INGENIERÍA} \\[0.3cm]
    {\large DEPARTAMENTO DE INGENIERÍA ELÉCTRICA} \\[1.5cm]

    % ----- Logo -----
    \includegraphics[width=3cm]{figures/institucion/escudo_udec.png} \\[2cm]
    
    % ----- Título del informe -----
    {\huge \textbf{\MainTitle}} \\[1.5cm]

    % ----- Información del autor -----
    {\Large \bfseries Grupo 17: } \\[5pt]
    {\Large\AuthorName} \\[2cm]

    {\Large \bfseries Informe 1} \\[1cm]
    
    {\large Profesor: \ProfessorName}

    \vfill
    % ----- Fecha en español -----
    Concepción, \today
\end{center}

\endgroup
\end{titlepage}
\begin{abstract}
    El presente proyecto aborda la problemática de la degradación térmica y pérdida de eficiencia en paneles fotovoltaicos, diseñando y validando una estrategia de enfriamiento activo por agua en circuito cerrado. El objetivo principal fue implementar un sistema de control en lazo cerrado capaz de mantener la temperatura operativa del panel bajo el límite de seguridad de \qty{55}{\celsius}, optimizando simultáneamente el consumo energético de los actuadores.

    La metodología consistió en la implementación de un algoritmo de control Proporcional ($P$) con compensación de punto de operación, gestionando de forma híbrida una bomba hidráulica y un ventilador mediante saturación de voltaje. A través de simulaciones dinámicas en MATLAB bajo escenarios de perturbación realistas (días despejados, nublados e intermitentes), se sintonizó una ganancia de $K_{c}=-0,25$, priorizando la estabilidad y la suavidad de actuación.

    Los resultados validaron la robustez del diseño, logrando mantener la temperatura máxima del panel en \qty{54,04}{\celsius} bajo condiciones de carga extrema, con un sobrepaso marginal de \qty{0,13}{\celsius}. El hallazgo más significativo fue la eficiencia energética del sistema: la estrategia propuesta redujo el consumo eléctrico diario en un \textbf{\qty{94,86}{\percent}} respecto a la operación convencional en lazo abierto, disminuyendo la demanda de \qty{0,720}{\kilo\watt{}\hour} a solo \qty{0,037}{\kilo\watt{}\hour}. Esto confirma la viabilidad técnica y económica de la modulación activa de potencia para la gestión térmica de sistemas fotovoltaicos.

\end{abstract}

% =================== ÍNDICE =======================
\newpage
\begin{spacing}{1}
    \setlength{\parskip}{0pt}

    % Configuración de profundidad
    \setcounter{secnumdepth}{3}
    \setcounter{tocdepth}{2}
    \pdfbookmark{\contentsname}{toc}

    % 1. Índice General (Va solo)
    \tableofcontents
    \newpage % Salto de página obligatorio después del índice general

    % 2. Índices de Figuras y Tablas (Juntos)
    \begingroup
    % Este comando mágico elimina el salto de página entre listas
    \let\clearpage\relax
    \let\cleardoublepage\relax

    % Imprimimos Figuras
    \listoffigures

    \vspace{2em} % Espacio manual para separar ambas listas

    % Imprimimos Tablas (aparecerá justo debajo)
    \listoftables
    \endgroup

\end{spacing}

% ========== CUERPO PRINCIPAL ==========
\newpage

\chapter{Introducción}

En la etapa anterior de este trabajo \cite{proyecto1}, se completó el desarrollo del modelado matemático que describe la dinámica térmica de un sistema de enfriamiento por agua en circuito cerrado, acoplado a un panel solar fotovoltaico. El sistema físico modelado consta de dos elementos activos manipulables: una bomba hidráulica encargada de la circulación del fluido refrigerante y un ventilador integrado a un intercambiador de calor, diseñado para disipar la energía térmica hacia el ambiente.

El resultado de dicha etapa fue un sistema de Ecuaciones Diferenciales Ordinarias (EDO) validado, que relaciona las entradas de tensión ($V_\text{bomb}, V_\text{vent}$) y las perturbaciones ambientales con la evolución temporal de la temperatura de operación del panel ($T_p$).

\section*{Planteamiento del Problema}

La eficiencia de conversión eléctrica de los paneles fotovoltaicos es sensible a la temperatura, cada grado sobre su temperatura de operación (\SI{25}{\degreeCelsius}) disminuye su rendimiento ligeramente. Sin un mecanismo de regulación activo, el panel queda expuesto a las variaciones estocásticas de la irradiancia solar y la temperatura ambiente.

El análisis preliminar del sistema (simulaciones en lazo abierto sin refrigeración) revela que, bajo condiciones de alta carga térmica, el panel tiende a alcanzar temperaturas de equilibrio cercanas a los \SI{71}{\degreeCelsius}. Este comportamiento supera ampliamente su estandar de operación, conllevando dos riesgos:
\begin{enumerate}
    \item Degradación acelerada de los materiales por estrés térmico.
    \item Pérdida de potencia útil generada.
\end{enumerate}

El desafío que aborda el presente proyecto es que la operación manual o fija del sistema de refrigeración es insuficiente. Operar los actuadores a máxima potencia de forma constante garantiza el enfriamiento, pero resulta \textbf{energéticamente ineficiente}, consumiendo gran parte de la energía que el propio panel genera. Por consiguiente, se hace indispensable diseñar un sistema de control en \textbf{lazo cerrado} que gestione este compromiso de forma autónoma.

\section*{Variables del Sistema de Control}

Para el diseño del esquema de control, se definen las siguientes variables sobre el modelo dinámico:

\begin{itemize}
    \item \textbf{Variable Controlada (Salida):} Temperatura superficial del panel solar ($T_p$).
    \item \textbf{Variables Manipuladas (Entradas):} Tensión aplicada a la bomba ($V_\text{bomb}$) y al ventilador ($V_\text{vent}$).
    \item \textbf{Perturbaciones:} Irradiancia solar ($G$), Temperatura ambiente ($T_\text{amb}$) y Velocidad del viento ($v_\text{vent}$).
\end{itemize}

\section*{Objetivos del Proyecto}

\subsection*{Objetivo General}
Diseñar e implementar un sistema de control realimentado (Controlador Proporcional) sobre el modelo dinámico del panel solar, capaz de satisfacer las especificaciones de respuesta transitoria y estacionaria definidas para la preservación del equipo y la eficiencia energética.

\subsection*{Especificaciones de Respuesta}
De acuerdo con los criterios de diseño postulados en la etapa de modelado, el controlador debe cumplir estrictamente con las siguientes especificaciones de desempeño:

\begin{enumerate}
    \item \textbf{Límite de Seguridad Térmica:} Mantener la temperatura superficial del panel por debajo de \SI{55}{\degreeCelsius} durante las horas de máxima irradiancia, con el fin de minimizar la degradación de los materiales.
    \item \textbf{Estabilidad del Régimen:} Garantizar un régimen térmico estable, evitando oscilaciones superiores a \SI{2}{\degreeCelsius} en la superficie del panel para prevenir fatiga por ciclo térmico.
    \item \textbf{Eficiencia Energética:} Optimizar el compromiso entre rendimiento térmico y consumo energético, mediante estrategias de refrigeración dinámica que ajusten su intensidad estrictamente de acuerdo con las condiciones ambientales.
\end{enumerate}

\subsection*{Objetivos Específicos}
Para dar cumplimiento a las especificaciones anteriores, se plantean las siguientes tareas:
\begin{itemize}
    \item Analizar el comportamiento dinámico en \textbf{lazo abierto} (ante entradas tipo escalón) para establecer la línea base de temperatura sin intervención.
    \item Programar un algoritmo de control Proporcional (P) a aplicar a las entradas manipulables.
    \item Sintonizar la ganancia ($K_c$) y el punto de operación (Offset) mediante variaciones de esta misma con el objetivo de reducir oscilaciones y asegurar el límite térmico.
    \item Evaluar cuantitativamente el desempeño del controlador mediante métricas de respuesta al escalón de referencia (sobrepaso, tiempo de estabilización y error de estado estacionario).
    \item Verificar la robustez del sistema simulando su comportamiento ante perfiles de perturbaciones realistas (días nublados e intermitentes).
\end{itemize}

\chapter{Metodología}

\section{Descripción del Sistema y Modelo Conceptual}

El sistema bajo estudio corresponde al modelo matemático desarrollado y validado en el proyecto pasado. Físicamente, se trata de una planta térmica compuesta por un panel fotovoltaico acoplado a un circuito de refrigeración activa por agua, el cual incluye una bomba de circulación y un intercambiador de calor (radiador) con ventilación forzada.

Para efectos del diseño de la estrategia de control, la física detallada del intercambio de calor se abstrae en un bloque dinámico o ``planta''. De acuerdo con la teoría de control clásica \cite{ogata2010modern}, el sistema se define por la relación causal entre sus variables:



\begin{itemize}
    \item \textbf{Entradas de Control ($u$):} Corresponden a las tensiones eléctricas aplicadas a los actuadores, específicamente la tensión de la bomba ($V_{bomb}$) y la del ventilador ($V_{vent}$). Estas son las variables que el controlador modificará para influir en el sistema.
    \item \textbf{Entradas de Perturbación ($d$):} Son variables exógenas que afectan la dinámica térmica pero no son manipulables, principalmente la irradiancia solar ($G$) y la temperatura ambiente ($T_{amb}$).
    \item \textbf{Variable de Salida ($y$):} Es la variable de interés a regular, en este caso, la temperatura superficial del panel ($T_p$).
\end{itemize}

La dinámica temporal del proceso está regida por el sistema de Ecuaciones Diferenciales Ordinarias (EDO) derivado de los balances de energía, cuya solución numérica se realiza en el entorno MATLAB.

\subsection*{Diagrama Esquemático en Lazo Abierto}

Con el fin de caracterizar la respuesta natural del sistema y cuantificar su inercia térmica antes de la implementación del controlador, se plantea un esquema de operación en \textbf{Lazo Abierto}. En esta configuración, no existe retroalimentación; la señal de control se genera de forma independiente al estado actual de la planta.

La Figura \ref{fig:diag:lazo-abierto} ilustra el modelo conceptual utilizado para esta etapa de diagnóstico:

\begin{figure}[H]
    \centering
    \includegraphics[width=\textwidth]{diagramas/lazo-abierto.pdf}
    \caption{Diagrama esquemático del proceso en lazo abierto con generador de señal tipo escalón.}
    \label{fig:diag:lazo-abierto}
\end{figure}

Como se observa en el diagrama, un \textbf{Generador de Señal} aplica un estímulo predefinido (tipo escalón) a los actuadores. Esta señal excita la planta térmica, la cual evoluciona bajo la influencia conjunta de la acción de control y las perturbaciones ambientales, resultando en la trayectoria de temperatura de salida.

\subsection*{Suposiciones y Limitaciones del Modelo}

Para el diseño y validación del controlador, se establecieron las siguientes simplificaciones sobre la física del sistema.

\begin{enumerate}

    \item \textbf{Dinámica de Actuadores Despreciable:}
          Se asume que la respuesta mecánica de los actuadores (tiempo que tarda el ventilador en alcanzar sus RPM o la bomba en establecer el flujo) es instantánea en comparación con la dinámica térmica del sistema.
          \textit{Justificación:} La constante de tiempo térmica del panel y el agua es del orden de minutos u horas, mientras que la constante de tiempo electromecánica de los motores suele trabajar en rangos desde milisegundos a segundos. Por tanto, se modela la relación Voltaje $\to$ Flujo como una ganancia estática pura.

    \item \textbf{Medición Ideal (Sensores sin Ruido):}
          El esquema de control asume que la variable realimentada ($T_p$) se obtiene de forma instantánea y libre de ruido estocástico o errores de cuantización.
          \textit{Justificación:} El objetivo principal es validar la lógica de control y la estabilidad térmica ante los escenarios de perturbación trabajados. En una implementación física futura, se requeriría una etapa de filtrado que no se contempla en esta etapa de diseño numérico.

\end{enumerate}


\section{Diseño de la Estrategia de Control en Lazo Cerrado}

Para cumplir con los objetivos de estabilización y seguridad térmica, pasa a un esquema de \textbf{Lazo Cerrado}. En esta configuración, el sistema mide continuamente la variable de salida ($T_p$) y la compara con una referencia deseada ($Ref$), generando una acción correctiva proporcional al error detectado \cite{ogata2010modern}.

El siguiente diagrama representa el esquema conceptual del sistema en lazo cerrado que será implementado:

\begin{figure}[H]
    \centering
    \includegraphics[width=\textwidth]{diagramas/lazo-cerrado.pdf}
    \caption{Diagrama esquemático del proceso en lazo cerrado}
    \label{fig:diag:lazo-cerrado}
\end{figure}

\subsection{Algoritmo de Control Proporcional}
Se usó la estrategia de control clásica de tipo Proporcional (P) con compensación de punto de operación (Offset). Estrategias de control más complejas no fueron usadas en el proyecto, sin embargo se deja abierta la posibilidad de implementarlos en un desarrollo futuro, considerado como mejoras de diseño.

La ley de control implementada en el simulador dinámico responde a la ecuación:

\begin{equation}
    u(t) = K_c \cdot (Ref - T_p(t)) + \text{Offset}
\end{equation}

Donde:
\begin{itemize}
    \item $u(t)$: Señal de control abstracta calculada por el algoritmo.
    \item $Ref$: Temperatura de referencia o Set-Point (Ej. \SI{55}{\degreeCelsius}).
    \item $T_p(t)$: Temperatura actual del panel (variable realimentada).
    \item $K_c$: Ganancia proporcional (Sensibilidad).
    \item $\text{Offset}$: Valor constante añadido para mantener el sistema en un punto de operación intermedio cuando el error es cero, evitando el apagado total de los actuadores.
\end{itemize}

Consideraciones:
\begin{itemize}
    \item \textbf{Signo de la Ganancia $K_c$}: Dado que el sistema busca generar refrigeración (generar tensión positiva) cuando la temperatura del panel excede la referencia (error negativo), la ganancia $K_c$ debe ser negativa para asegurar una respuesta adecuada del controlador.
    \item \textbf{Valor del Offset} Se fijó un valor para el Offset de \SI{5}{\volt}, fijado experimentalmente con el objetivo de evitar activaciones bruscas en momentos tempranos del día. Este valor puede ser ajustado en función de las condiciones ambientales y los objetivos de eficiencia energética.
\end{itemize}

\subsection{Gestión de Actuadores}

El sistema físico cuenta con dos entradas manipulables ($V_{bomb}, V_{vent}$) pero el proyecto se estructuró de forma que se genera una única señal de mando $u(t)$ la cual será aplicada en cada actuador de igual manera.

De forma experimental se obtuvo que la bomba de agua tiene relativamente los mismos efectos cuando esta funciona a su voltaje nominal (\SI{24}{\volt}) que a un voltaje reducido de \SI{3}{\volt}. Por lo tanto, se decidió que la tensión aplicada a la bomba será limitada a \SI{3}{\volt}.

Posteriormente, para evitar irregularidades, la señal $u(t)$ se limita al rango operativo de los actuadores, es decir, entre \SI{0}{\volt} y \SI{12}{\volt} para el ventilador y entre \SI{0}{\volt} para la bomba de agua. De esta forma, valores negativos de $u$, que concluirían en un intento de ''calentar'' el panel aplicando tensión negativa, se ajusta a \SI{0}{\volt} (apagado), mientras que valores superiores a \SI{12}{\volt} (asociados a un error que el sistema no es capaz corregir) se limitan a \SI{12}{\volt} para el ventilador, y \SI{3}{\volt} para la bomba (máximo posible).

Esta estrategia híbrida permite mantener el circuito de agua activo con un consumo mucho más reducido, delegando la carga fuerte de disipación al ventilador solo cuando es necesario.

Dada estas consideraciones se señalan las entradas que finalmente son aplicadas en los actuadores:

\begin{equation*}
    \left\{
    \begin{gathered}
        V_{bomb}(t) = \min\left(\max\left(u(t), 0\right), 3\right) \\
        V_{vent}(t) = \min\left(\max\left(u(t), 0\right), 12\right)
    \end{gathered}
    \right.
\end{equation*}
\vspace{5pt}
con $\displaystyle u(t) = K_c \cdot (Ref - T_p(t)) + \text{Offset}$

\section{Métricas de Evaluación} \label{sec:metricas}

A continuación se definen las métricas utilizadas para cuantificar el desempeño del sistema. Estas se calculan numéricamente a partir de los vectores de tiempo ($t$) y temperatura ($T_p$) obtenidos en la simulación.

\subsection{Temporales}
Para aplicar correctamente estas métricas, se efectuan los cálculos y mediciones sobre un entorno de simulación controlada, dicho entorno se caracteriza en la sección

En dicho entorno se aplica un escalón en la referencia de temperatura ($Ref$) en la salida del controlador y se observa la respuesta del sistema.

\subsubsection{Tiempo de subida ($T_r$)}

Se define como el tiempo necesario para que la respuesta del sistema recorra del \SI{10}{\percent} al \SI{90}{\percent} de su variación total real. Dado que el sistema es sobreamortiguado y no oscila, esta métrica es el indicador principal de la velocidad de reacción térmica.
\begin{equation}
    T_r = t_{90\%} - t_{10\%}
\end{equation}

\subsubsection{Tiempo de estabilización ($T_s$)}
Corresponde al tiempo de asentamiento o llegada. Se define como el instante en que la temperatura entra por primera vez en una banda de tolerancia del \SI{1}{\percent} respecto al valor final alcanzado ($T_{final}$).
\begin{equation}
    T_s = \min \{ t \mid |T_p(t) - T_\text{final}| \leq 0.01 \cdot \Delta T_\text{total}\}
\end{equation}

\subsubsection{Sobrepaso ($M_p$)}
Es la diferencia máxima entre el valor pico de la respuesta y el valor final estabilizado. En sistemas de control de temperatura, se busca que este valor sea nulo o despreciable para evitar estrés térmico en los materiales.
\begin{equation}
    M_p = \max(T_p) - T_{final}
\end{equation}

\subsubsection{Error de estado estacionario ($e_{ss}$)}
Es la diferencia permanente entre la referencia deseada ($Ref$) y el valor real obtenido una vez que el sistema ha alcanzado el equilibrio térmico ($t \to \infty$).
\begin{equation}
    e_{ss} = Ref - T_{final}
\end{equation}

\subsection{Energéticas (Consumo de Energía)}

Para evaluar el desempeño energético del sistema con controlador implementado, es necesario cuantificar el consumo total de energía eléctrica por parte de los actuadores durante un ciclo operativo completo (un día simulado).

En el Proyecto~1 se tomó un modelo específico de bomba de agua, mientras que para el ventilador se tomaron valores arbitrarios, en este proyecto se tomará una aproximación para la potencia consumida por cada actuador en función de la tensión aplicada a estos asumiendo que la corriente consumida es proporcional a la tensión aplicada (modelo lineal simplificado), en consecuencia, la potencia instantánea será proporcional al cuadrado de la tensión. La energía total acumulada ($E_{total}$) se calcula integrando numéricamente estas potencias en el tiempo:

\begin{equation}
    E_{total}(t) = \int_{0}^{t} \left( \frac{V_{bomb}(\tau)^2}{R_{bomb}} + \frac{V_{vent}(\tau)^2}{R_{vent}} \right)\,d\tau \quad [\si{\joule}]
\end{equation}

Donde $R_{bomb}$ y $R_{vent}$ son las resistencias equivalentes estimadas de los actuadores. Esta métrica penaliza el uso de voltajes altos, reflejando fielmente el costo operativo real.

\begin{itemize}
    \item $\mathbf{R_\text{vent}}$: Los ventiladores trabajan comunmente potencias entre \SI{5}{\watt} y \SI{15}{\watt}, para el ventilador se tomó una aproximación de \SI{12}{\watt} de potencia nomilan, de esta forma podemos realizar aproximaciones de la resistencia equivalente:
          \begin{equation*}
              P_\text{max, vent} = \frac{(V_\text{nom})^2}{R_\text{vent}} \implies \SI{12}{\watt} = \frac{(\SI{12}{\volt})^2}{R_\text{vent}} \implies \boxed{R_\text{vent} = \SI{12}{\ohm}}
          \end{equation*}
    \item $\mathbf{R_\text{bomb}}$: La bomba de agua utilizada corresponde a la bomba \textbf{\textit{Iwaki RD-20}} \Cite{datasheet_bomba}, la cual funciona a \SI{60}{\watt} a su tensión nominal de \SI{24}{\volt}, con estos datos se obtiene su resistencia equivalente:
          \begin{equation*}
              P_\text{max, bomb} = \frac{(V_\text{nom})^2}{R_\text{bomb}} \implies \SI{60}{\watt} = \frac{(\SI{24}{\watt})^2}{R_\text{bomb}} \implies \boxed{R_\text{bomb} = \frac{48}{5}~\si{\ohm}}
          \end{equation*}
\end{itemize}

con estos valores será posible cuantificar la energía consumida por cada actuador sobre el sistema tanto en lazo abierto como cerrado.

\section{Simulación}

Para la resolución computacional de este sistema de ecuaciones diferenciales no lineales, se mantuvo el uso del solucionador \texttt{ode45} de MATLAB, el cual implementa un método de Runge-Kutta de orden 4(5) con paso adaptativo \cite{shampine1997matlab}, la versión de MATLAB utilizada fue la \textit{R2025b}.

\subsection{Escenarios de Simulación} \label{ssec:escenarions-de-simulacion}

Para evaluar el desempeño del sistema de control propuesto, se diseñaron diversos escenarios de simulación que replican condiciones operativas típicas y extremas. Estos escenarios corresponden a variaciones en la irradiancia solar y en la temperatura ambiente los cuales constituirán diferentes perfiles de carga térmica en el panel fotovoltaico.

Todos los escenarios a continuación parten del mismo escenario de temperatura ambiente e irradiancia solar y velocidad del viento definidos en el proyecto anterior, el cual simula un día típico soleado y despejado.

\subsubsection*{Escenario Base}

Esta configuración de perturbaciones fue desarrollado en el proyecto anterior resultando en las siguientes expresiones matemáticas asociadas a la temperatura ambiente, irradiancia solar y velocidad del viento:

\begin{equation}
    T_\text{amb}\left(t_h = \frac{t}{3600}\right) =
    \left\{
    \begin{aligned}
        T_N + (T_s - T_N) exp\left(-\frac{b((t_h+24)-t_\text{set})}{24-(t_\text{set}-t_\text{min})}\right), & \quad 0\leq t_h < t_\text{min}               \\[15pt]
        (T_x-T_N)\sin\left(\frac{\pi (t_h-t_\text{min})}{(t_\text{set}-t_\text{min})+2\alpha}\right) + T_N, & \quad t_\text{min}\leq t_h \leq t_\text{set} \\[15pt]
        T_N + (T_s - T_N) exp\left(-\frac{b(t_h-t_\text{set})}{24-(t_\text{set}-t_\text{min})}\right),      & \quad t_\text{set}< t_h\leq24
    \end{aligned}
    \right.
    \label{eq:temperatura-ambiente}
\end{equation}


\begin{equation}
    G\left(t_h = \frac{t}{3600}\right) =
    \begin{cases}
        \hspace*{1.5cm} 0,                                                                      & t_h\leq t_\text{amanecer}              \\
        G_\text{max}\sin\left(\frac{\pi(t_h-t_\text{rise})}{t_\text{set}-t_\text{rise}}\right), & t_\text{rise}\leq t_h\leq t_\text{set} \\
        \hspace*{1.5cm} 0,                                                                      & ,t_\text{set}\leq t_h
    \end{cases}
    \label{eq:irradiancia-solar}
\end{equation}

\begin{equation}
    v_\text{vien}(t) = 5+3\sin\left(\frac{2\pi}{24\cdot3600}\,t \right) , \quad \text{ con $t$ en } \si{\second}
    \label{eq:velocidad-viento}
\end{equation}

Respecto a el uso de la velocidad del viento, esta se mantuvo igual para todos los escenarios de simulación, dado que su impacto en la dinámica térmica del sistema es relativamente menor en comparación con la irradiancia solar y la temperatura ambiente.

A continuación se presentarán los ajustes aplicados a estas expresiones base para generar los diferentes escenarios de simulación.

\subsubsection{Día Despejado} \label{sssec:dia-despejado}

Este escenario corresponde directamente al caso base original usado en el proyecto anterior, los parámetros usados en las ecuaciones \ref{eq:temperatura-ambiente}, \ref{eq:irradiancia-solar} y son los siguientes:


\begin{table}[H]
    \centering
    \begin{tabular}{ll}
        \hline
        \textbf{Parámetro} & \textbf{Valor}                     \\ \hline
        $T_N$              & \SI{10}{\degreeCelsius}            \\
        $T_x$              & \SI{35.2}{\degreeCelsius}          \\
        $T_s$              & \SI{25.1018}{\degreeCelsius}       \\
        $t_\text{min}$     & \SI{6.5}{\hour}                    \\
        $t_\text{set}$     & \SI{20.5}{\hour}                   \\
        $b$                & 2.5                                \\
        $\alpha$           & 1.8                                \\
        $G_\text{max}$     & \SI{1000}{\watt\per\meter\squared} \\
        $t_\text{rise}$    & \SI{6.5}{\hour}
        \\ \hline
    \end{tabular}
    \caption{Valores usados para la generación de $T_\text{amb}$ y $G$ para el día soleado}
    \label{tab:valores-escenario-1}
\end{table}

\subsubsection{Día Nublado}

En este escenario, para simular condiciones de nubosidad, se reduce significativamente la irradiancia solar incidente sobre el panel fotovoltaico. Para ello, se ajusta el valor máximo de irradiancia $G_\text{max}$ en la ecuación \ref{eq:irradiancia-solar} a un valor reducido de \SI{500}{\watt\per\meter\squared}, además se modifican algunos valores en la tabla \ref{tab:valores-escenario-1} para reflejar un día más frío y con menor temperatura máxima:

\begin{table}[H]
    \centering
    \begin{tabular}{ll}
        \hline
        \textbf{Parámetro} & \textbf{Valor}                    \\ \hline
        $T_N$              & \SI{12}{\degreeCelsius}           \\
        $T_x$              & \SI{25}{\degreeCelsius}           \\
        $T_s$              & \SI{20}{\degreeCelsius}           \\
        $t_\text{min}$     & \SI{6.5}{\hour}                   \\
        $t_\text{set}$     & \SI{20.5}{\hour}                  \\
        $b$                & 2.5                               \\
        $\alpha$           & 1.8                               \\
        $G_\text{max}$     & \SI{500}{\watt\per\meter\squared} \\
        $t_\text{rise}$    & \SI{6.5}{\hour}
        \\ \hline
    \end{tabular}
    \caption{Valores usados para la generación de $T_\text{amb}$ y $G$ para el día nublado}
    \label{tab:valores-escenario-2}
\end{table}


\subsubsection{Día Parcialmente Nublado (Intermitente)}


A diferencia de los escenarios anteriores que presentan variaciones lentas y continuas, el perfil intermitente (o parcialmente nublado) se construye para someter al controlador a perturbaciones de alta frecuencia. Este escenario se deriva del modelo base de día despejado aplicando dos mecanismos de transformación distintos, según la naturaleza física de la variable:

\begin{enumerate}
    \item \textbf{Bloqueo de Irradiancia (Instantáneo):} Se modeló el paso de formaciones nubosas que atenúan la radiación solar de forma abrupta (ondas cuadradas). El modelo considera la superposición de nubes densas y nubes pasajeras rápidas.

    \item \textbf{Enfriamiento Ambiental (Inercial):} Se modeló la caída de temperatura del aire asociada al oscurecimiento, aplicando un desfase temporal y una suavización de la curva para representar la inercia térmica atmosférica.
\end{enumerate}

La formulación matemática detallada de estas funciones de perturbación, así como los parámetros temporales de periodicidad y factores de atenuación utilizados, se encuentran descritos en el Apéndice \ref{app:A} resultando en las siguientes expresiones para la generación de $G$ y $T_\text{amb}$:

\subsubsection{Entorno Controlado para Análisis de Métricas} \label{sssec:entorno-controlado}

Para aplicar correctamente las métricas de desempeño temporal definidas en la sección anterior, se efectúan los cálculos y mediciones sobre un entorno de simulación controlada. Este entorno se caracteriza por mantener condiciones ambientales constantes durante el periodo de análisis, eliminando las variaciones diurnas y perturbaciones externas.

A continuación se detallan las condiciones específicas del entorno controlado:

\begin{itemize}
    \item \textbf{Irradiancia Solar:} Se fija un valor de irradiancia solar constante de \SI{800}{\watt\per\meter\squared}.
    \item \textbf{Temperatura Ambiente: }Se fija un valor de \SI{30}{\degreeCelsius}
    \item \textbf{Velocidad del Viento: }Se fija una velocidad constante de \SI{5}{\meter\per\second}.
\end{itemize}

En lo referente al escalón de referencia de temperatura ($Ref$), el sistema se inicia con condiciones iniciales de \SI{30}{\degreeCelsius} con una referencia de \SI{45}{\degreeCelsius} la cual se espera una hora hasta que los valores se estabilizan, posteriormente se produce un escalón de referencia establecido en \SI{55}{\degreeCelsius}, generando una variación de \SI{15}{\degreeCelsius} que permite observar la respuesta transitoria del sistema bajo condiciones controladas, aplicando las métricas definidas previamente.

\subsection{Pruebas/Simulaciones a Realizar}

Con el fin de evaluar exhaustivamente el desempeño del sistema de control propuesto, se planifican las siguientes simulaciones y pruebas:



\begin{itemize}
    \item \textbf{Simulación en Lazo Abierto:} Se ejecutará una simulación del sistema sin la implementación del controlador, aplicando las condiciones de los tres escenarios definidos (Día Soleado, Día Nublado, Día Parcialmente Nublado). El objetivo es caracterizar la respuesta natural del sistema y establecer una línea base para comparación.

    \item \textbf{Ajuste de Ganancia Proporcional ($K_c$):} Se realizarán múltiples simulaciones iterativas variando el valor de la ganancia proporcional $K_c$ sobre el mismo escenario base (El día soleado y despejado). El objetivo es identificar el valor óptimo que más se adapte a las especificación de respuesta definidas previamente.

    \item \textbf{Simulación en Lazo Cerrado:} Se realizará una simulación con el controlador implementado, evaluando su desempeño bajo las mismas condiciones ambientales de los tres escenarios utilizados en las simulaciones en lazo abierto. Esto permitirá analizar la efectividad del controlador en la regulación de la temperatura.

    \item \textbf{Simulación en Entorno Controlado:} Se llevará a cabo una simulación específica en el entorno controlado descrito en la Sección \ref{subsecc:entorno-controlado}. Esta prueba se centrará en la evaluación de las métricas temporales de desempeño del sistema, permitiendo un análisis sin la presencia de perturbaciones.

\end{itemize}
\chapter{Resultados y Análisis}

\section{Caracterización de la Respuesta en Lazo Abierto}

Para dar cumplimiento al objetivo de visualizar la dinámica natural del proceso, se implementó un esquema de \textbf{Lazo Abierto} incorporando un generador de señal tipo escalón aplicado a las variables manipulables ($V_{bomb}, V_{vent}$).

En esta prueba, el sistema es sometido a las perturbaciones estandarizadas del ``Día Soleado'' (irradiancia y temperatura ambiente variables descritas en la Sección \ref{ssec:escenarios-de-simulacion}), mientras que los actuadores son activados mediante una señal fija (escalón de tensión) durante las horas de operación, sin recibir retroalimentación de la temperatura real del panel. Esta simulación fue la más efectiva en reducción de la temperatura del panel respecto a las entradas utilizadas en el Proyecto 1 \cite{proyecto1}.

La Figura~\ref{fig:plot:lazo_abierto} muestra la respuesta temporal obtenida bajo estas condiciones.

\begin{figure}[H]
      \centering
      \includegraphics[width=1\textwidth]{plots/temp-panel-actuadores-sin-control.pdf}
      \caption{Respuesta dinámica del sistema en Lazo Abierto ante una entrada tipo escalón en horas de mayor carga térmica con perturbaciones de día despejado.}
      \label{fig:plot:lazo_abierto}
\end{figure}

\subsection*{Análisis de la Respuesta Dinámica}

Del comportamiento observado en la gráfica, se desprenden las siguientes conclusiones sobre la planta:

\begin{enumerate}
      \item \textbf{Capacidad de Enfriamiento Suficiente:} Se observa que, al aplicar la señal de control fija (máxima capacidad), la temperatura máxima alcanzada es de aproximadamente \SI{51}{\degreeCelsius}. Este valor se encuentra por debajo del límite crítico de seguridad (\SI{55}{\degreeCelsius}) definido en las especificaciones de respuesta, lo que indica que los actuadores dimensionados tienen la capacidad física suficiente para disipar la carga térmica máxima del sistema si operan a plena potencia.

      \item \textbf{Dependencia de las Perturbaciones:} A pesar de que la temperatura se mantiene en rangos seguros, la curva de salida sigue parcialmente la forma de la irradiancia solar (sube al mediodía y baja al atardecer). Esto evidencia que, en lazo abierto, el sistema es incapaz de rechazar perturbaciones completamente debido a la magnitud de la inercia térmica del ambiente, por lo cual la temperatura flota libremente en función del clima, sin mantener un punto de operación constante (Set-Point) debido al escalón.

      \item \textbf{Descenso Drástico de Temperatura en Activación:} Al momento de la activación del sistema el panel sufre un descenso de temperatura cercano a \qty{10}{\degreeCelsius}, esto en el día a día conlleva un estrés térmico constante a los materiales del panel reduciendo su vida útil.

      \item \textbf{Ineficiencia Energética (el desafío de consumo):} La señal escalón implica que los actuadores están operando a un nivel de tensión constante incluso en momentos donde la carga térmica es baja (mañana y tarde). Esto representa un desperdicio de energía significativo, ya que se está aplicando un esfuerzo de enfriamiento máximo cuando no es estrictamente necesario.
\end{enumerate}

\paragraph{Conclusión} Si bien la operación en lazo abierto puede satisfacer el requisito de seguridad térmica bajo estas condiciones ($\approx \SI{51}{\degreeCelsius}$), carece de capacidad de regulación y eficiencia. Esto justifica la implementación de una estrategia de control en \textbf{Lazo Cerrado} que module el esfuerzo de los actuadores para mantener la temperatura estable y minimizar el consumo.

\section{Análisis de Estabilidad y Sintonización del Controlador}

Una vez caracterizada la planta, se procedió al diseño del controlador en lazo cerrado. Para determinar la ganancia proporcional ($K_c$) adecuada, se realizó un análisis de sensibilidad sometiendo al sistema al escenario de carga máxima (Día Soleado) variando la magnitud de la ganancia en el conjunto de valores $\{-0.25, -0.5, -1.0, -2.0\}$, junto con un valor de referencia para el controlador de $Ref = \qty{45}{\degreeCelsius}$ y un valor de Offset de \qty{5}{\volt}.

El objetivo de esta comparativa es encontrar el compromiso óptimo entre la velocidad de reducción de temperatura y la estabilidad de la señal de control enviada a los actuadores.

La Figura~\ref{fig:plot:comparativa-kc} presenta la superposición de las respuestas temporales y el esfuerzo de control resultante para los distintos valores de ganancia probados.

\begin{figure}[H]
      \centering
      \caption{\textbf{(a)} Temperatura (\unit{\degreeCelsius}) del panel solar. \textbf{(b)} Tensión (\unit{\volt}) aplicada en cada actuador.}
      \begin{subfigure}[t]{0.45\textwidth}
            \includegraphics[width=\textwidth]{plots/comparativa-kc-temp-panel.pdf}
            \caption{}
            \label{fig:plot:comparativa-kc-a}
      \end{subfigure}
      \hfil
      \begin{subfigure}[t]{0.45\textwidth}
            \includegraphics[width=\textwidth]{plots/comparativa-kc-volt.pdf}
            \caption{}
            \label{fig:plot:comparativa-kc-b}
      \end{subfigure}
      \label{fig:plot:comparativa-kc}
\end{figure}

A partir de la visualización del comportamiento dinámico, se observan dos fenómenos contrapuestos:

\begin{enumerate}
      \item \textbf{Comportamiento con Altas Ganancias ($K_c = -1.0, -2.0$):}
            Si bien las ganancias de mayor magnitud logran acercar la temperatura a la referencia más rápidamente, se observa en el gráfico inferior de la Figura~\ref{fig:plot:comparativa-kc-b} que la señal de voltaje entra en una zona de \textbf{saturación oscilatoria} o comportamiento agresivo.
            Este fenómeno provoca cambios bruscos en la velocidad del ventilador, lo cual es indeseable mecánicamente y no se traduce en una mejora térmica significativa debido a la lenta inercia térmica del agua.

      \item \textbf{Comportamiento con Bajas Ganancias ($K_c = -0.25$):}
            Con una ganancia conservadora de $-0.25$, la respuesta de temperatura es más suave con oscilaciones constantes pero menores. Aunque presenta temperatura máxima ligeramente mayor a los demás, logra mantener la temperatura por debajo del límite de seguridad de \SI{55}{\degreeCelsius}.
            Lo más relevante es que la señal de control (Voltaje) evoluciona suavemente, siguiendo la curva de irradiancia sin saltos abruptos, lo que maximiza la vida útil de los actuadores y de los materiales del panel solar.
\end{enumerate}

\paragraph{Selección de Parámetros} Basado en este análisis, se selecciona $K_c = -0.25$ como la ganancia definitiva para el controlador. Esta sintonía prioriza la robustez y la suavidad de la actuación por sobre la velocidad de respuesta, dado que el sistema térmico es inherentemente lento y el criterio principal es la seguridad operativa ($T < \SI{55}{\degreeCelsius}$) junto a la eficiencia energética más que la precisión de seguimiento al Offset fijado.

\section{Evaluación de Desempeño Transitorio}

Habiendo seleccionado la ganancia definitiva ($K_c = -0.25$), se procede a cuantificar el desempeño dinámico del sistema mediante las métricas definidas en la Sección~\ref{sec:metricas}. Cabe aclarar que este análisis no fue posible realizarlo en la etapa de sintonización debido a la presencia de perturbaciones variables (irradiancia y temperatura ambiente) que distorsionan la respuesta natural del sistema.

La evaluación se realiza bajo el \textbf{Entorno Controlado} descrito en la Sección~\ref{sssec:entorno-controlado}, el cual aísla el sistema de perturbaciones variables manteniendo condiciones constantes: irradiancia solar de \qty{800}{\watt\per\meter\squared}, temperatura ambiente de \qty{30}{\degreeCelsius} y velocidad del viento de \qty{5}{\meter\per\second}.

El protocolo de prueba consiste en una secuencia de dos etapas:
\begin{enumerate}
      \item \textbf{Pre-estabilización:} Se inicia la simulación con el panel a temperatura ambiente (\qty{30}{\degreeCelsius}) y el controlador operando con una referencia inicial de $Ref = \qty{45}{\degreeCelsius}$. Se permite que el sistema evolucione durante una hora ($t=0$ a $t=\qty{3600}{\second}$) hasta alcanzar el equilibrio.
      \item \textbf{Prueba de Escalón (Step Test):} En el instante $t=\qty{1}{\hour}$, se aplica un cambio tipo escalón en la referencia, aumentando su valor a \qty{55}{\degreeCelsius}.
\end{enumerate}

El análisis de métricas temporales ($T_r, T_s, M_p, e_{ss}$) se aplica específicamente sobre la respuesta del sistema durante esta segunda etapa (transición de \qty{45}{\degreeCelsius} a \qty{55}{\degreeCelsius}).

La Figura~\ref{fig:plot:entorno-controlado} ilustra la respuesta temporal del sistema durante la prueba. Se observa la evolución de la temperatura (línea azul) intentando alcanzar la nueva referencia (línea roja discontinua) a partir del segundo cambio de escalón.

\begin{figure}[H]
      \centering
      \caption{Respuesta transitoria del sistema controlado ante un escalón de referencia de \qty{45}{\degreeCelsius} a \qty{55}{\degreeCelsius}.}
      \label{fig:plot:entorno-controlado}
      \begin{subfigure}[b]{0.45\textwidth}
            \includegraphics[width=\textwidth]{plots/entorno-controlado-temp-panel.pdf}
            \caption{Respuesta transitoria: Temperatura (\unit{\degreeCelsius}) del panel solar.}
            \label{fig:plot:entorno-controlado-a}
      \end{subfigure}
      \begin{subfigure}[b]{0.45\textwidth}
            \includegraphics[width=\textwidth]{plots/entorno-controlado-volt.pdf}
            \caption{Esfuerzo de control: Tensión (\unit{\volt}) aplicada a los actuadores.}
            \label{fig:plot:entorno-controlado-b}
      \end{subfigure}
\end{figure}

A partir de esta respuesta, se calcularon las métricas de desempeño definidas en la metodología. Los resultados se presentan en la Tabla~\ref{tab:metricas}.

\begin{table}[H]
      \centering
      \begin{tabular}{|l|c|c|}
            \hline
            \textbf{Métrica}               & \textbf{Símbolo} & \textbf{Valor Obtenido}    \\
            \hline
            Tiempo de Subida               & $T_r$            & \qty{16.81}{\minute}       \\
            Tiempo de Estabilización (5\%) & $T_s$            & \qty{21.56}{\minute}       \\
            Sobrepaso Máximo               & $M_p$            & \qty{0.13}{\degreeCelsius} \\
            Error de Estado Estacionario   & $e_{ss}$         & \qty{5.54}{\degreeCelsius} \\
            \hline
      \end{tabular}
      \caption{Resumen de métricas de desempeño temporal para $K_c = -0.25$.}
      \label{tab:metricas}
\end{table}

\subsection*{Análisis de Desempeño}

A partir de los datos obtenidos en la Tabla~\ref{tab:metricas}, se desprenden las siguientes conclusiones técnicas sobre la calidad del control implementado:

\begin{enumerate}
      \item \textbf{Respuesta Subamortiguada con Oscilación Residual:}
            Se registra un sobrepaso máximo de \qty{0.13}{\degreeCelsius} y una oscilación sostenida de baja amplitud en torno al valor final. Esto clasifica el comportamiento como \textbf{subamortiguado}.
            Si bien existe oscilación perpetua (ciclo límite), su amplitud es marginal ($< 0.3\%$ de la referencia). En términos prácticos, esta magnitud es despreciable frente a la inercia térmica del panel y no representa un riesgo de fatiga mecánica para los actuadores ni estrés térmico para el material.

      \item \textbf{Dinámica Temporal ($T_r, T_s$):}
            Los tiempos de respuesta (subida en $\approx \qty{17}{\minute}$ y estabilización en $\approx \qty{22}{\minute}$) confirman la alta inercia térmica de la masa de agua. El controlador respeta la dinámica natural de la planta sin forzar una actuación agresiva que podría saturar innecesariamente la bomba, logrando un compromiso adecuado entre velocidad de enfriamiento y suavidad operativa.

      \item \textbf{Margen de Seguridad Pasivo ($e_{ss}$):}
            Se observa un Error de Estado Estacionario de \qty{5.54}{\degreeCelsius}. Al utilizar un controlador Proporcional sin acción integral, este error es matemáticamente esperado.
            Es fundamental notar la dirección del error: la temperatura se estabiliza en $\approx \qty{49.5}{\degreeCelsius}$ para una referencia de \qty{55}{\degreeCelsius}. Esto actúa como un \textbf{margen de seguridad} de casi \qty{6}{\degreeCelsius} a favor del sistema, garantizando que la temperatura operativa se mantenga lejos del límite crítico de degradación. El error determinado está directamente asociado a la elección de un valor de Offset.
\end{enumerate}

\section{Validación ante Escenarios de Perturbación Dinámica}

Una vez verificado el desempeño transitorio en condiciones ideales, se sometió el sistema de control a los perfiles de perturbación de 24 horas definidos en la Sección \ref{ssec:escenarios-de-simulacion}. El objetivo de esta sección es validar la capacidad del controlador para mantener la temperatura del panel dentro de los límites de seguridad ($T_p < \SI{55}{\degreeCelsius}$) bajo distintas condiciones operativas.

A continuación, se analiza la respuesta del sistema controlado ($K_c = -0.25$) frente a los tres escenarios climáticos planteados.

\subsection{Escenario 1: Carga Máxima (Día Despejado)}

Se simuló el comportamiento bajo el perfil de ``Día Soleado'', el cual representa la condición más exigente para los actuadores (irradiancia pico de \qty{1000}{\watt\per\meter\squared}).

La Figura \ref{fig:plot:dia-despejado} muestra la evolución de la temperatura del panel y la tensión aplicada a los actuadores durante las 24 horas.

\begin{figure}[H]
      \centering
      \begin{subfigure}[b]{0.45\textwidth}
            \caption{}
            \label{fig:plot:dia-despejado-a}
            \includegraphics[width=\textwidth]{plots/dia-despejado-temp-panel.pdf}
      \end{subfigure}
      \hfill
      \begin{subfigure}[b]{0.45\textwidth}
            \caption{}
            \label{fig:plot:dia-despejado-b}
            \includegraphics[width=\textwidth]{plots/dia-despejado-volt.pdf}
      \end{subfigure}
      \caption{\textbf{(a)} Temperatura (\unit{\degreeCelsius}) del panel. \textbf{(b)} Tensión (\unit{\volt}) aplicada a los actuadores.}
      \label{fig:plot:dia-despejado}
\end{figure}
\paragraph{Análisis}
\begin{itemize}
      \item \textbf{Seguridad Térmica:} Durante las horas de máxima radiación, la temperatura se estabiliza por debajo de los \qty{50}{\degreeCelsius}, cumpliendo holgadamente el requisito de seguridad ($< \SI{55}{\degreeCelsius}$) en el peor caso posible.
      \item \textbf{Actuación Progresiva:} La señal de control evoluciona suavemente siguiendo la curva solar, sin presentar saturaciones bruscas que dañen los motores.
      \item \textbf{Bajo Consumo Energético:} El bajo consumo del ventilador confirma la elección de la ganancia a pesar del ligero aumento en la temperatura máxima, ya que la modulación suave del voltaje eficiencia energética destacable.
\end{itemize}

\subsection{Escenario 2: Carga Reducida (Día Nublado)}

Este escenario simula un día con baja irradiancia ($G_{max} = \qty{500}{\watt\per\meter\squared}$), útil para verificar que el sistema no desperdicie energía sobre-enfriando el panel cuando no es necesario.

La Figura~\ref{fig:plot:dia-nublado} presenta los resultados obtenidos.

\begin{figure}[H]
      \centering
      \begin{subfigure}[b]{0.45\textwidth}
            \caption{}
            \label{fig:plot:dia-nublado-a}
            \includegraphics[width=\textwidth]{plots/dia-nublado-temp-panel.pdf}
      \end{subfigure}
      \hfill
      \begin{subfigure}[b]{0.45\textwidth}
            \caption{}
            \label{fig:plot:dia-nublado-b}
            \includegraphics[width=\textwidth]{plots/dia-nublado-volt.pdf}
      \end{subfigure}
      \caption{\textbf{(a)} Temperatura (\unit{\degreeCelsius}) del panel. \textbf{(b)} Tensión (\unit{\volt}) aplicada a los actuadores.}
      \label{fig:plot:dia-nublado}
\end{figure}

\paragraph{Análisis}
\begin{itemize}
      \item \textbf{Adaptabilidad Energética:} Se observa que, debido a la menor carga térmica, el esfuerzo de control (voltaje) es significativamente menor que en el día despejado.
      \item \textbf{Prevención de Sobre-enfriamiento:} El sistema mantiene la temperatura cercana a la referencia sin forzar los actuadores al máximo, validando que el algoritmo proporcional ajusta su intensidad según la demanda real.
      \item \textbf{Oscilaciones Dentro del Rango Permitido:} Si bien la temperatura del panel presenta pequeñas oscilaciones residuales de alta frecuencia en las horas de mayor carga, la amplitud de estas variaciones se mantiene significativamente por debajo del umbral de tolerancia de \qty{2}{\degreeCelsius} establecido en las especificaciones de respuesta, garantizando la estabilidad y previniendo la fatiga por ciclo térmico.
\end{itemize}

\subsection{Escenario 3: Día Parcialmente Nublado (Día Intermitente)}

Finalmente, se evaluó la respuesta ante cambios abruptos de nubosidad para probar la velocidad de reacción del lazo cerrado.

\begin{figure}[H]
      \centering
      \begin{subfigure}[b]{0.45\textwidth}
            \caption{}
            \label{fig:plot:dia-intermitente-a}
            \includegraphics[width=\textwidth]{plots/dia-intermitente-temp-panel.pdf}
      \end{subfigure}
      \hfill
      \begin{subfigure}[b]{0.45\textwidth}
            \caption{}
            \label{fig:plot:dia-intermitente-b}
            \includegraphics[width=\textwidth]{plots/dia-intermitente-volt.pdf}
      \end{subfigure}
      \caption{\textbf{(a)} Temperatura (\unit{\degreeCelsius}) del panel. \textbf{(b)} Tensión (\unit{\volt}) aplicada a los actuadores.}
      \label{fig:plot:dia-intermitente}
\end{figure}

\paragraph{Análisis}
\begin{itemize}
      \item \textbf{Rechazo a Perturbaciones:} Al producirse una caída abrupta de irradiancia (paso de nube), el controlador reduce rápidamente el voltaje de los actuadores, evitando el gasto innecesario de energía.
      \item \textbf{Estabilidad:} A pesar de la naturaleza cuadrada de la perturbación solar, la temperatura del panel varía suavemente gracias a la inercia térmica, manteniéndose siempre en rangos seguros y estables.
\end{itemize}

\section{Análisis de Eficiencia Energética}

Finalmente, se evaluó la viabilidad económica del diseño cuantificando el costo energético. Para ello, se calculó el \textbf{Esfuerzo de Control Acumulado} (en \unit{\kilo\watt{}\hour}) comparando el desempeño de la estrategia propuesta frente a la operación convencional y analizando su adaptabilidad ante distintos climas.

\subsection{Comparación: Lazo Abierto vs. Lazo Cerrado}

Se contrastó el consumo del controlador propuesto ($K_c=-0.25$) frente a la operación en Lazo Abierto con señal fija ($V_{vent}=\qty{12}{\volt}, V_{bomb}=\qty{3}{\volt}$) bajo el escenario de mayor demanda (Día Soleado), los resultados de estas entradas en el panel solar fueron mostrados en la Figura~\ref{fig:diag:lazo-abierto}. En las Figuras \ref{fig:plot:energia-consumida-lazo-abierto} y \ref{fig:plot:energia-consumida-lazo-cerrado} se observa la energía acomulada a lo largo del día mencionado.

\begin{figure}[H]
      \centering
      \includegraphics[width=0.8\textwidth]{plots/energia-lazo-abierto.pdf}
      \caption{Energía Consumida por el sistema de refrigeración en Lazo Abierto.}
      \label{fig:plot:energia-consumida-lazo-abierto}
\end{figure}

\begin{figure}[H]
      \centering
      \includegraphics[width=0.8\textwidth]{plots/energia-dia-despejado.pdf}

      \caption{Energía Consumida por el sistema de refrigeración en Lazo Cerrado.}
      \label{fig:plot:energia-consumida-lazo-cerrado}
\end{figure}

La Tabla~\ref{tab:ahorro_global} resume el impacto del control en el ahorro total.

\begin{table}[H]
      \centering
      \begin{tabular}{|l|c|c|c|}
            \hline
            \textbf{Estrategia}      & \textbf{Energía Total}         & \textbf{Ahorro Absoluto}       & \textbf{\% Ahorro}    \\
            \hline
            Lazo Abierto (Base)      & \qty{0.720}{\kilo\watt{}\hour} & -                              & -                     \\
            Lazo Cerrado (Propuesto) & \qty{0.037}{\kilo\watt{}\hour} & \qty{0.063}{\kilo\watt{}\hour} & \qty{94.86}{\percent} \\
            \hline
      \end{tabular}
      \caption{Cuantificación del ahorro energético diario.}
      \label{tab:ahorro_global}
\end{table}

\subsection{Desglose por Escenarios Climáticos}

Además del ahorro base, se verificó la capacidad del controlador para reducir el consumo en días de menor carga térmica. La Tabla~\ref{tab:res:desglose_escenarios} detalla el aporte de cada actuador según el clima.

\begin{table}[H]
      \centering
      \begin{tabular}{|l|c|c|c|}
            \hline
            \textbf{Escenario} & \textbf{E. Ventilador (\unit{\kilo\watt{}\hour})} & \textbf{E. Bomba (\unit{\kilo\watt{}\hour})} & \textbf{Total Diario (\unit{\kilo\watt{}\hour})} \\
            \hline
            Día Soleado (Máx)  & 0.027                                             & 0.010                                        & 0.037                                            \\
            Día Nublado (Mín)  & 0.003                                             & 0.004                                        & 0.007                                            \\
            Día Intermitente   & 0.019                                             & 0.009                                        & 0.027                                            \\
            \hline
      \end{tabular}
      \caption{Consumo energético (\unit{\kilo\watt{}\hour}) desagregado por actuador y condición climática.}
      \label{tab:res:desglose_escenarios}
\end{table}

\subsection*{Discusión de Resultados}

\begin{itemize}
      \item \textbf{Ahorro Significativo:} La implementación del lazo cerrado genera un ahorro del \qty{94.76}{\percent} respecto a la operación fija. Esto valida que modular la potencia es económicamente superior a mantener el sistema encendido al máximo, incluso en el día más caluroso.

      \item \textbf{Adaptabilidad Automática:} Los datos de la Tabla~\ref{tab:res:desglose_escenarios} demuestran que el consumo disminuye considerablemente en los días Nublado e Intermitente. El controlador "siente" la menor carga térmica y reduce el esfuerzo del ventilador automáticamente, sin intervención humana.

      \item \textbf{Validación Estrategia SIDO:} En todos los escenarios, el consumo de la bomba se mantiene bajo y estable, validando que limitar su tensión a \qty{3}{\volt} fue una decisión de diseño correcta para minimizar el costo base operativo.
\end{itemize}
\chapter{Conclusiones}

El presente proyecto logró diseñar, simular y validar exitosamente una estrategia de control para la gestión térmica eficiente de un panel fotovoltaico. A partir del análisis de los resultados obtenidos, se establecen las siguientes conclusiones principales:

\begin{enumerate}
      \item \textbf{Cumplimiento de Seguridad Térmica:}
            Se verificó que el controlador Proporcional implementado ($K_c = -0.25$) es capaz de mantener la temperatura del panel consistentemente por debajo del límite crítico de \qty{55}{\degreeCelsius}. En el escenario más exigente (Día Soleado), la temperatura máxima registrada fue inferior a \qty{50}{\degreeCelsius}, garantizando la integridad física del equipo y evitando la degradación prematura de los materiales.

      \item \textbf{Eficiencia Energética Superior:}
            La evaluación comparativa demostró que la estrategia de control propuesta reduce el consumo energético en un \qty{94.8}{\percent} respecto a una operación convencional de lazo abierto (potencia fija). El sistema demostró capacidad para adaptar el esfuerzo de control a la demanda real, reduciendo el consumo diario de \qty{0.720}{\kilo\watt{}\hour} a apenas \qty{0.037}{\kilo\watt{}\hour}, validando la viabilidad económica de la solución.

      \item \textbf{Validación de la Estrategia:}
            La decisión de diseño de saturar la bomba hidráulica a un voltaje de mantenimiento (\qty{3}{\volt}) y delegar la modulación dinámica al ventilador resultó exitosa. El análisis desagregado reveló que esta configuración minimiza el consumo base del sistema sin comprometer la capacidad de enfriamiento, confirmando que operar la bomba a su tensión nominal (\qty{24}{\volt}) habría sido energéticamente ineficiente e innecesario.

      \item \textbf{Robustez ante Perturbaciones:}
            Las simulaciones bajo condiciones intermitentes confirmaron que el lazo de control posee la velocidad de reacción suficiente para adaptarse a cambios bruscos de nubosidad. El sistema ajusta rápidamente el voltaje de los actuadores ante caídas de irradiancia, evitando el sobre-enfriamiento y optimizando el uso de recursos.

      \item \textbf{Estabilidad Dinámica:}
            A pesar de utilizar un controlador puramente proporcional, el sistema exhibe un comportamiento estable. El error de estado estacionario resultante actúa favorablemente como un margen de seguridad operativo.
\end{enumerate}

Como trabajo futuro, se sugiere la implementación de estrategias de control más avanzadas, específicamente la incorporación de una \textbf{Acción Derivativa (Control PD o PID)}. La adición del término derivativo permitiría anticipar la tasa de cambio de la temperatura, mejorando la amortiguación de las oscilaciones residuales observadas, mientras que el término integral habilitaría una acción de control más enérgica para alcanzar el Set-Point con mayor rapidez y precisión, sin comprometer la estabilidad del sistema.

\paragraph{Modificación en Parámetros de Diseño} Debido a la suficiencia de \qty{3}{\volt} de la bomba de agua para mantener la temperatura bajo control, es posible obtener el caudal que esta genera a dicha tensión y ajustar los parámetros y modelo de la bomba usada en consecuencia, ya que por lo general estos dispositivos no son capaces de trabajar a fracciones tan bajas de su tensión nominal sin perder eficiencia. Esto permitiría optimizar aún más el consumo energético del sistema.

% ========== APÉNDICES ===========
\newpage
\appendix

\chapter{Modelado Matemático de Perturbaciones Intermitentes} \label{app:A}

En este apéndice se detalla la formulación analítica utilizada para construir las señales de perturbación correspondientes al escenario de "Día Parcialmente Nublado" descrito en la Metodología.

\section{Función de Irradiancia Solar}

El perfil de irradiancia intermitente se modela como una modulación de la irradiancia base de cielo despejado ($G_{base}$), afectada por un coeficiente de transmitancia variable $\beta(t)$ que representa el bloqueo óptico de las nubes:

\begin{equation}
    G_{int}(t) = G_{base}(t) \cdot \beta(t)
\end{equation}

El coeficiente $\beta(t)$ resulta de la superposición de dos trenes de pulsos independientes, $\beta_1(t)$ y $\beta_2(t)$, que simulan diferentes tipos de formaciones nubosas:

\begin{equation}
    \beta(t) = \beta_1(t) \cdot \beta_2(t)
\end{equation}

Cada función $\beta_i(t)$ se define como una señal periódica rectangular utilizando la operación módulo para determinar los intervalos de oscurecimiento:

\begin{equation}
    \beta_i(t) =
    \begin{cases}
        \alpha_i & \text{si } \left( t \pmod{T_i} \right) < \Delta t_i \\
        1        & \text{en otro caso}
    \end{cases}
\end{equation}

Donde:
\begin{itemize}
    \item $T_i$: Periodo de recurrencia de la nube.
    \item $\Delta t_i$: Duración del bloqueo solar.
    \item $\alpha_i$: Factor de transmitancia (1 = transparente, 0 = opaco).
\end{itemize}

Los parámetros utilizados en la simulación para cada componente son:

\begin{table}[H]
    \centering
    \begin{tabular}{lccc}
        \hline
        \textbf{Componente}       & \textbf{Periodo ($T$)} & \textbf{Duración ($\Delta t$)} & \textbf{Transmitancia ($\alpha$)} \\
        \hline
        Nube Densa ($\beta_1$)    & \qty{2.5}{h}           & \qty{25}{min}                  & 0.3                               \\
        Nube Pasajera ($\beta_2$) & \qty{1.1}{h}           & \qty{8}{min}                   & 0.6                               \\
        \hline
    \end{tabular}
    \caption{Parámetros de modulación para la irradiancia intermitente.}
\end{table}

\section{Función de Temperatura Ambiente con Inercia}

A diferencia de la irradiancia, la temperatura del aire no cambia instantáneamente. Se modela mediante una reducción suave sobre la temperatura base ($T_{base}$), introduciendo una dinámica inercial:

\begin{equation}
    T_{amb, int}(t) = T_{base}(t) - \Delta T_{inercial}(t)
\end{equation}

El término de reducción $\Delta T_{inercial}(t)$ se activa únicamente durante las horas de sol y se calcula mediante una función cosenoidal invertida y normalizada:

\begin{equation}
    \Delta T_{inercial}(t) = \Delta T_{max} \cdot \left( \frac{1 - \cos\left( \frac{2\pi}{T_1}(t - \tau) \right)}{2} \right)
\end{equation}

Donde los parámetros físicos son:
\begin{itemize}
    \item $\Delta T_{max} = \qty{4}{\degreeCelsius}$: Máxima caída de temperatura del aire debido a la sombra.
    \item $T_1 = \qty{2.5}{h}$: Frecuencia sincronizada con la nubosidad principal.
    \item $\tau = \qty{45}{min}$: \textbf{Retardo por inercia térmica}. Representa el tiempo que tarda la masa de aire en comenzar a enfriarse tras el cambio en la radiación.
\end{itemize}

Esta formulación asegura que la reducción inicie suavemente desde cero (continuidad de clase $C^1$), evitando saltos abruptos no físicos en la simulación.

% ========== BIBLIOGRAFÍA ==========
\newpage
\pdfbookmark{Bibliografía}{bibliografia}
\printbibliography[
    heading=bibintoc, % Añade la entrada al índice
    title={Bibliografía}
]



\end{document}