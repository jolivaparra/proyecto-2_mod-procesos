\chapter{Resultados y Análisis}

\section{Caracterización de la Respuesta en Lazo Abierto}

Para dar cumplimiento al objetivo de visualizar la dinámica natural del proceso, se implementó un esquema de \textbf{Lazo Abierto} incorporando un generador de señal tipo escalón aplicado a las variables manipulables ($V_{bomb}, V_{vent}$).

En esta prueba, el sistema es sometido a las perturbaciones estandarizadas del ``Día Soleado'' (irradiancia y temperatura ambiente variables descritas en la Sección \ref{subsec:escenarions-de-simulacion}), mientras que los actuadores son activados mediante una señal fija (escalón de tensión) durante las horas de operación, sin recibir retroalimentación de la temperatura real del panel.

La Figura \ref{fig:res:lazo_abierto} muestra la respuesta temporal obtenida bajo estas condiciones.

\begin{figure}[H]
    \centering
    % Asegúrate de que el nombre del archivo coincida con tu imagen guardada
    % \includegraphics[width=0.9\textwidth]{figures/resultados/lazo_abierto_escalon.png}
    \caption{Respuesta dinámica del sistema en Lazo Abierto ante una entrada tipo escalón en horas de mayor carga térmica con perturbaciones de día despejado.}
    \label{fig:res:lazo_abierto}
\end{figure}

\subsection*{Análisis de la Respuesta Dinámica}

Del comportamiento observado en la gráfica, se desprenden las siguientes conclusiones sobre la planta:

\begin{enumerate}
    \item \textbf{Capacidad de Enfriamiento Suficiente:} Se observa que, al aplicar la señal de control fija (máxima capacidad), la temperatura máxima alcanzada es de aproximadamente \SI{53}{\degreeCelsius}. Este valor se encuentra por debajo del límite crítico de seguridad (\SI{55}{\degreeCelsius}), lo que indica que los actuadores dimensionados tienen la capacidad física suficiente para disipar la carga térmica máxima del sistema si operan a plena potencia.

    \item \textbf{Dependencia de las Perturbaciones (El Desafío de Regulación):} A pesar de que la temperatura se mantiene en rangos seguros, la curva de salida sigue la forma de la irradiancia solar (sube al mediodía y baja al atardecer). Esto evidencia que, en lazo abierto, el sistema es incapaz de rechazar perturbaciones completamente; la temperatura flota libremente en función del clima, sin mantener un punto de operación constante (Set-Point).

    \item \textbf{Ineficiencia Energética (El Desafío de Consumo):} La señal escalón implica que los actuadores están operando a un nivel de tensión constante incluso en momentos donde la carga térmica es baja (mañana y tarde). Esto representa un desperdicio de energía significativo, ya que se está aplicando un esfuerzo de enfriamiento máximo cuando no es estrictamente necesario.
\end{enumerate}

\textbf{Conclusión:} Si bien la operación en lazo abierto puede satisfacer el requisito de seguridad térmica bajo estas condiciones ($\approx \SI{53}{\degreeCelsius}$), carece de capacidad de regulación y eficiencia. Esto justifica la implementación de una estrategia de control en \textbf{Lazo Cerrado} que module el esfuerzo de los actuadores para mantener la temperatura estable y minimizar el consumo.

\section{Sintonización de la Ganancia del Controlador}

\section{Evaluación de Desempeño Transitorio}

\section{Validación ante Escenarios de Perturbación Dinámica}

\section{Análisis de la Eficiencia Energética}


