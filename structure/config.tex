\setlength{\headheight}{14.5pt} % Ya corregido.



% filas más compactas y fuente más pequeña en todas las tablas
\AtBeginEnvironment{tabular}{\small\renewcommand{\arraystretch}{0.9}}
\AtBeginEnvironment{tabularx}{\small\renewcommand{\arraystretch}{0.9}}
\setlength{\tabcolsep}{4pt}

% INTERLINEADO Y PÁRRAFOS
\onehalfspacing                     % Interlineado 1.5
\setlength{\parskip}{0.4cm}         % espacio entre párrafos
\setlength{\parindent}{1cm}         % sangría

% ENCABEZADOS Y PIE DE PÁGINA
\fancypagestyle{plain}{
    \fancyhf{} % Limpia todos losñ encabezados y pies existentes
    \fancyfoot[C]{\thepage}
    \renewcommand{\headrulewidth}{0pt} % Asegura que no haya línea en el encabezado
    \renewcommand{\footrulewidth}{0pt} % Asegura que no haya línea en el pie de página
}

% COMANDOS ÚTILES DEL INFORME
\newcommand{\AuthorName}{Joshua S. Oliva Parra}
\newcommand{\ProfessorName}{Juan Pablo Segovia Vera}
\newcommand{\MainTitle}{Diseño de Control y Optimización Energética para Sistema de Enfriamiento de Paneles Fotovoltaicosgit }

% EXTRAS
\sisetup{
    per-mode = symbol,              % usa "/" en lugar de exponentes negativos
    inter-unit-product = \cdot,     % punto medio para multiplicar unidades
}
\numberwithin{equation}{chapter}
\numberwithin{figure}{chapter}
\numberwithin{table}{chapter}

% === CONFIGURACIÓN DE TÍTULOS DE ÍNDICES (PAQUETE TOCLOFT) ===


% 1. Reducir espacio vertical ANTES del título (Sube el título)
\setlength{\cftbeforeloftitleskip}{-2em}
\setlength{\cftbeforelottitleskip}{-2em}
\setlength{\cftbeforetoctitleskip}{-2em} % También lo aplicamos al índice general si quieres

% 2. Reducir espacio vertical DESPUÉS del título
\setlength{\cftafterloftitleskip}{10pt}
\setlength{\cftafterlottitleskip}{10pt}
\setlength{\cftaftertoctitleskip}{10pt}

% 3. Cambiar tamaño de fuente (De \Huge a \Large)
\renewcommand{\cftloftitlefont}{\Large\bfseries\sffamily}
\renewcommand{\cftlottitlefont}{\Large\bfseries\sffamily}
\renewcommand{\cfttoctitlefont}{\Large\bfseries\sffamily}

% Insertar Códigos
% Colores personalizados (opcional)
\definecolor{miblue}{rgb}{0,0,1}
\definecolor{migray}{rgb}{0.4,0.4,0.4}
\definecolor{miteal}{rgb}{0.0,0.5,0.5}

% Configuración general de listings para MATLAB
\lstset{
language=Matlab,
inputencoding=utf8,
extendedchars=true,
literate={ñ}{{\~n}}1
{Ñ}{{\~N}}1
{á}{{\'a}}1
{é}{{\'e}}1
{í}{{\'i}}1
{ó}{{\'o}}1
{ú}{{\'u}}1
{Á}{{\'A}}1
{É}{{\'E}}1
{Í}{{\'I}}1
{Ó}{{\'O}}1
{Ú}{{\'U}}1
{°}{{$^{\circ}$}}1
{·}{{$\cdot$}}1
{Ω}{{$\Omega$}}1  % <--- ESTA ES LA SOLUCIÓN CLAVE
{µ}{{$\mu$}}1      % Recomendado: Para micro (ej. µF)
{π}{{$\pi$}}1,     % Recomendado: Para pi
backgroundcolor=\color{white},
basicstyle=\ttfamily\scriptsize,
numbers=left,
numberstyle=\tiny,
numbersep=4pt,
xleftmargin=0.5em,
xrightmargin=0.5em,
breaklines=true,
frame=single,
captionpos=b,
keywordstyle=\color{miblue},
commentstyle=\color{migray},
stringstyle=\color{miteal},
showstringspaces=false
}