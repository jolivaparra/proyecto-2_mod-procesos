
% ===========================
% PAQUETES BÁSICOS
% ===========================
\usepackage[spanish]{babel}             % Idioma español
\usepackage[T1]{fontenc}                % Codificación de salida
\usepackage[utf8]{inputenc}             % Codificación de entrada
\usepackage{csquotes}
\usepackage[letterpaper,
    top=2.5cm,
    bottom=2cm,
    left=2.5cm,
    right=2cm,
    headsep=0.3cm,
    footskip=1.2cm]{geometry}

\usepackage{newtxtext,newtxmath}        % Fuente Times New Roman
\usepackage{setspace}                   % Interlineado
\usepackage{multicol}
\let\Bbbk\relax                % Evitar conflicto con newtxmath

% ===========================
% FIGURAS Y TABLAS
% ===========================
\usepackage{graphicx}                   % Imágenes
\graphicspath{{figures/}}
\usepackage{float}                      % Posicionamiento [H]
\usepackage[table]{xcolor}              % Colores
\usepackage[skip=4pt, labelfont=bf,
    justification=centering,
    margin=2cm]{caption}
\usepackage{subcaption}                 % Leyendas de subfiguras
\usepackage{wrapfig}


% ===========================
% MATEMÁTICAS Y CIENCIAS
% ===========================
\usepackage{amssymb, mathtools, esint, bm}  % Símbolos y entornos matemáticos
\usepackage{siunitx}                    % Unidades físicas

% ===========================
% TÍTULOS Y PÁGINA
% ===========================
\usepackage{titlesec}                   % Formato de capítulos/secciones
\usepackage{fancyhdr}                   % Encabezados y pies de página

% ===========================
% HIPERVÍNCULOS Y CITAS
% ===========================
\usepackage[
    backend=biber,         % Motor de procesamiento recomendado
    style=ieee,            % Estilo de citas numérico (ajustado de ieeetr)
    natbib=true            % Permite usar \citep y \citet
]{biblatex}
\addbibresource{references.bib}

\usepackage[bookmarks=true,
    bookmarksnumbered,
    colorlinks=false,
    hidelinks,
    breaklinks]{hyperref}


% =========================
% OTROS
% =========================
\usepackage{listings}
% Asegúrate de tener este paquete para el símbolo de grados
\usepackage{textcomp}

\usepackage[titles]{tocloft} % Configuración de títulos